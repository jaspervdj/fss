\documentclass[a4paper,10pt]{article}

\usepackage[dutch]{babel}
\usepackage{a4wide}

\title{Project Formele Systeemmodellering voor Software}
\author{Bruno Corijn, Jasper Van der Jeugt, Toon Willems}
\date{\today}

\begin{document}

\maketitle

\section{Deel 1: ACL2}

Voor de priority-queue zijn er drie versies terug te vinden. Oorspronkelijk
waren we begonnen met de queue te implementeren als een binomial heap. Dit heeft
als voordeel dat het veel sneller is dan de andere alternatieven. Het grote
nadeel was echter de moeilijke formele verificatie van deze datastructuur.

Hierna hebben we geprobeerd de queue te implementeren als eenvoudige linked
list. De eigenschappen hiervan zijn min of meer tegenovergesteld aan die van de
binomial heap: de correctheid was makkelijker te bewijzen maar het was echter
ook zeer ineffici\"ent voor de gebruiker, waardoor we ook dit niet verder hebben
uitgewerkt.

Uiteindelijk hebben we gekozen voor een binaire boom. In het slechtste geval zal
ook deze datastructuur traag werken, gezien er niet wordt geherbalanceerd maar
gemiddeld zal deze implementatie sneller zijn dan de linked list.

De finale implementatie kan teruggevonden worden in \texttt{binary-tree.lisp}.
De twee andere (onafgewerkte) implementaties bevinden zich in
\texttt{linked-list.lisp} en \texttt{binomial-heap.lisp}.

\subsection{Interface}

We bespreken eerst kort de "publieke API" van de queue. Nieuwe wachtrijen
aanmaken:

\begin{itemize}
    \item \texttt{queue-empty} maakt een lege wachtrij aan.
    \item \texttt{queue-singleton k v} maakt een wachtrij aan met een element
    met als waarde \texttt{v} en als prioriteit \texttt{k}.
\end{itemize}

Eigenschappen van queues opvragen:

\begin{itemize}
    \item \texttt{queue-null queue} kijkt of de gegeven wachtrij leeg is.
    \item \texttt{queue-size queue} bepaalt het aantal elementen in de wachtrij.
    \item \texttt{queue-find-min-value queue} geeft de waarde terug die bij de
    hoogste prioriteit hoort.
\end{itemize}

Wachtrijen bewerken:

\begin{itemize}
    \item \texttt{queue-insert k v queue} voegt een element met als waarde
    \texttt{v} en als prioriteit \texttt{k} toe aan de wachtrij.
    \item \texttt{queue-delete-min queue} verwijdert het element met de hoogste
    prioriteit.
    \item \texttt{queue-merge q1 q2} voegt de wachtrijen \texttt{q1} en
    \texttt{q2} tesamen.
    \item \texttt{queue-change-priority k v queue} verandert de prioriteit van
    de elementen met als waarde \texttt{v} naar \texttt{k}.
\end{itemize}

\subsection{Formele specificaties}

De specificaties zijn terug te vinden als theorems, die gebruik maken van enkele
hulp functies.

Deze functies controleren de volgende zaken: of alle elementen in de queue een
kleinere prioriteit hebben dan een gegeven \texttt{x}, of alle elementen in de
queue een grotere of gelijke prioriteit hebben aan een gegeven \texttt{x}, of er
een element met prioriteit \texttt{k} en value \texttt{v} terug te vinden is in
een gegeven queue.

Het belangrijkste hulppredicaat is \texttt{queue-valid}, dit gaat na of de
ordening van de gehele boom correct is. Dit is dan ook een nodige voorwaarde bij
de meeste theorema's.

We specifi\"eren dat na enkele mogelijke update (een element toevoegen,
verwijderen...) \texttt{queue\-valid} behouden blijft. Op die manier kunnen we
zeggen dat \texttt{queue-valid} altijd klopt als de wachtrij is opgebouwd via de
functies uit de publieke API.

In de code is er bij elk theorema commentaar voorzien zodat het duidelijk
zou moeten zijn waartoe het onderstaand theorema dient. Hieronder zullen we
in grote lijnen de theorema's overlopen.

De theorema's kunnen verdeeld worden aan de hand van de functie waarover
ze gaan. De eerste paar theorema's bewijzen het gedrag en de geldigheid van een
queue die enkel uit een singleton bestaat. Vervolgens hebben we meerdere
theorema's die het gedrag van de \texttt{queue-insert} functie bekijken. Er
wordt aangetoond dat deze functie effectief een element toevoegt, er geen
andere elementen worden toegevoegd (of verwijderd) buiten het meegegeven
element en dat de correctheid van de boom bewaard blijft.

Er zijn enkele hulptheorema's te vinden die onder andere de transitiviteit van
\texttt{queue-all-get} aantonen en dat alle element groter zijn dan het minimum.
Daarna volgen er meeredere theorema's over eigenschappen van
\texttt{queue-delete-min}. We kunnen hiermee aantonen dat de functie enkel en
alleen het kleinste element zal verwijderen, de overblijvende elementen allemaal
groter zijn dan het verwijderde element, de correctheid bewaard blijft en de
boom krimpt hierdoor.

De laatste theorema's controleren de geldigheid en gedrag van \texttt{queue-merge}
en \texttt{queue-change\-priority}.

\subsection{Verificatie}

De output van de automatische verificatie van de implementatie als binaire boom
is te vinden als bijlage aan ons verslag.

\end{document}
