\documentclass[a4paper,10pt]{article}

\usepackage[dutch]{babel}
\usepackage{a4wide}

\title{Project Formele Systeemmodellering voor Software}
\author{Bruno Corijn, Jasper Van der Jeugt, Toon Willems}
\date{\today}

\begin{document}

\maketitle

\section{Deel 2: TLA/TLC}

Het oorspronkelijk protocol is terug te vinden in de Trein.* files, die we
eerst zullen bespreken. Vervolgens kijken we naar wat er verbeterd moet worden
, deze veranderingen zijn terug te vinden in de BeterTrein.* bestanden.

\subsection{Originele specificatie en vereisten}

In de eerste versie zijn enkel het protocol en de vereisten gespecifi\"eerd in de
opgave ge\"implementeerd. Hiervoor gebruiken we de volgende variabelen:
\begin{itemize}
    \item \texttt{deuren} geeft aan op de conducteur de passagiersdeuren heeft
    gesloten.
    \item \texttt{conductdeurDeur} geeft aan of de conducteur zijn eigen deur
    heeft gesloten.
    \item \texttt{ac} geeft aan of het Action Completed signaal is gegeven.
    \item \texttt{seinlicht} bevat de huidige kleur van het seinlicht. Dit kan
    rood of wit zijn.
    \item \texttt{vertrek} geeft aan of de trein vertrokken is.
\end{itemize}

In de initi\"ele toestand zijn al deze variabelen 0, buiten het seinlicht dat op
rood zal staan. Deze veriabelen zullen van waarde veranderen aan de hand van de
volgende acties:
\begin{itemize}
    \item \texttt{Fluitsignaal} Vereist dat de deuren open zijn en het seinlicht
    op rood staat. Hierdoor worden de deuren gesloten.
    \item \texttt{Action} Als de deuren dicht zijn, kan het AC signaal gegeven
    worden.
    \item \texttt{Seinlicht} Als het AC signaal gegeven is, kan het seinlicht van
    kleur veranderen naar wit.
    \item \texttt{Vertrek} Indien het seinlicht op wit staat kan de trein
    vertrekken.
    \item \texttt{Reset} Zet na vertrek van de trein alle variabelen terug naar
    hun oorspronkelijke staat.
    \item \texttt{Deur} Als de trein vertrokken is kan de conducteur zijn deur
    sluiten.
\end{itemize}

Daarnaast hebben we ook nog specificaties \texttt{Next} dat altijd een van de
bovenstaande acties zal zijn, \texttt{Live} die ervoor zorgt dat de
variabelen correct blijven en \texttt{Spec} die specifi\"eert dat het programma
altijd start uit de initi\"ele toestand, via \texttt{Next} in de huidige toestand
kan belanden en dat \texttt{Live} altijd geldt.

Als laatste hebben we de veiligheid- en fairnessvereisten van dit systeem die
ge\"impliceerd worden door \texttt{Spec}. Dit zijn de volgende:

\begin{itemize}
    \item \texttt{VertrekNaAc} Na het geven van het AC signaal zal het seinlicht
    verspringen van kleur.
    \item \texttt{RoodSeinlichtDefault} In de begintoestand is het seinlicht
    altijd rood.
    \item \texttt{Fairness} Als het seinlicht wit is, zal de trein vertrekken.
    \item \texttt{Veiligheid} De trein mag niet vertrekken zonder de conducteur.
    \item \texttt{OoitVertrek} De trein zal ooit vertrekken.
\end{itemize}

\subsection{Problemen}

Door de automatische modelchecker zien we dat de veiligheidsvereiste kan
geschonden worden door het bestaande protocol, dit zullen we oplossen in de
verbeterde versie.

\subsection{Verbeteringen}

Om te voorkomen dat de trein kan vertrekken zonder de conducteur is er een extra
regel toegevoegd aan de \texttt{Vertrek} actie, die vereist dat de deur van de
conducteur gesloten is alvorens de trein kan vertrekken. Hierdoor is er wel
voldaan aan de veiligheidsvereiste

\subsection{Verificatie}

De output en configuratie van de automatische verificatie van beide versies zijn
te vinden als bijlages aan ons verslag.

\end{document}
